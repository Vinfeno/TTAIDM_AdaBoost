AdaBoost zeichnet sich unter anderem dadurch aus, dass der Algorithms benutzerfreundlich und flexibel ist, wodurch er leicht an
spezifische Probleme angepasst werden kann. Ein weiterer Vorteil ist, dass AdaBoost automatisch wichtige Features identifiziert und diese
entsprechend im Ensemble gewichtet. Letztlich neigt AdaBoost zudem weniger zum Overfitting, da nur schwache Lerner genutzt werden, deren
Kapazität zu gering ist, um die Daten \glqq auswendig\grqq~zu lernen.\\\\
Jedoch ist AdaBoost anfällig für Ausreißer und verrauschte Daten, weil sich die adaptive Gewichtsanpassung mit jeder Iteration komulativ auf falsch
klassifizierte Daten fokussiert und so versucht, schwer klassifizierbare Daten zu berücksichtigen. Zudem kann das Training auf großen Datensätzen
sehr zeitintensiv sein, da oftmals für hunderte bis tausende Iterationen trainiert wird, wobei jedes mal eine große Menge von Lernern trainiert und evaluiert
werden müssen. Zuletzt ist AdaBoost hauptsächlich für binäre Klassifikation ausgelegt und somit nicht für Regression einsetzbar und
nur durch Erweiterungen für Multiklassen-Probleme genutzt werden kann.