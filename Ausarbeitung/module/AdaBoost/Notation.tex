\begin{itemize}
    \item Sei $\mathcal{X}$ die Menge der Features mit $\left|\mathcal{X}\right| = n$
          (Anzahl der Features) und $\mathcal{Y}$ die Menge der Labels, die gelernt werden sollen.
          Dabei ist $\mathcal{Y}=\{-1, +1\}$ bei binärer Klassifikation. Ein Trainingdatensatz $D$ besteht aus $m$ Einträgen,
          welche Features mit Labels verbinden:
          $$
              D=\{(\boldsymbol{x}_i, y_i)\},~i=1, \dots, m
          $$
    \item Nach dem Training auf $D$ wird ein Lernalgorithmus $\mathcal{L}$ eine Hypothese bzw. einen Klassifizierer
          $h$ zurück geben, der von $\mathcal{X}$ nach $\mathcal{Y}$ abbildet.
          \begin{align*}
              h:\mathcal{X} \rightarrow \mathcal{Y}, h(\boldsymbol{x}) = y
          \end{align*}
    \item $T$ ist die Anzahl der gewünschten Trainingsiterationen.
    \item Bei jeder Iteration $t=1, \dots,T$ wird ein Datensatz $\mathcal{D}_t$ von $D$ abgeleitet. Dabei wird
          jeder Datenpunkt mit einem Gewicht $\mathcal{D}_t(i)$ mit $i=1, \dots, m$ erweitert.
\end{itemize}
