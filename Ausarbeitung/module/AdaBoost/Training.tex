Trainiere für $t=1,\dots,T$ Iterationen mehrere schwache Lerner unter berücksichtigung der aktuellen Gewichtung.
$$
    h = \mathcal{L}(D, w_t)
$$
(Hinweis: Abhängig vom konkreten Lernalgorithmus wird ggf. zusätzlich zum gewichteten Datensatz $\mathcal{D}_t$
auch der gesamte Datensatz $D$ benötigt. Daher werden formal beide übegeben.)\\
Ziel ist es, den besten Lerner in der Iteration auszuwählen und diesen dem Ensemble hinzuzufügen. Dazu werden
die Gewichte der falsch vorhergesagten Datenpunkte summiert:
$$
    \varepsilon_j = \sum_{i=1}^m w^{(t)}_i\cdot I\left(y_i \neq h_j\left(\boldsymbol{x}_i\right)\right)
$$
$I$ ist die Indikatorfunktion, die $1$ zurückgibt, wenn $y_i\neq h_t(\boldsymbol{x_i})$ erfüllt ist, und
sonst $0$. Das bedeutet, sie gibt $1$ für jeden falsch vorhergesagten Datenpunkt zurück. $\varepsilon_t$
stellt die gewichtete Summe aller falschen Vorhersagen des $t$-ten Modells dar. Das Modell
mit dem geringsten Fehler wird in jeder Iteration ausgewählt. Ein Fehler von $1$ zeigt, dass alle
Vorhersagen falsch und $0$ bedeutet, dass alle korrekt sind.
Bei einem Fehler von $0.5$ sind entsprechend die Hälfte der Vorhersagen richtig. \\\\
Wenn $\varepsilon_t > 0.5$, sind die Vorhersagen also schlechter
als zufälliges Raten und der Algorithmus (in der ursprünglichen Version) wird gestoppt,
da weitere Lerner das Modell nicht verbessern würden.\\\\
In der Praxis werden aber meist mehrere Modelle pro Iteration trainiert, wobei nicht bei einem Fehler von $\varepsilon_t>0.5$
gestoppt wird. Schwache Lerner werden für jedes Feature trainiert und beide Polaritäten betrachtet (damit also insgesamt
$2n$ Modelle). Wenn ein Modell z.B. nur zu 40\% korrekt vorhersagt, wird durch Umkehrung der Polarität eine 60\%ige Genauigkeit erreicht. So gibt es für
jede Iteration stets eine Auswahl von Modellen, die als $h_t$ gewählt werden können.