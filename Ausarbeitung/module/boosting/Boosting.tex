Boosting ist eine Ensemble-Technik im Machine Learning, die mehrere sog. \emph{\glqq schwache Lerner\grqq} kombiniert,
um ein präziseres Gesamtmodell zu erstellen. Als \emph{schwacher Lerner} wird ein Modell bezeichnet, das nur einen geringen
Zusammenhang in den Daten lernen kann und dessen Vorhersagen daher nur etwas besser als zufälliges Raten sind.
In jeder Iteration wird ein solches Modell trainiert, das falsche
Vorhersagen der vorherigen Modelle korrigiert, indem es die Gewichtung der Trainingsdaten anpasst.
Das Ziel ist es, den systematischen Fehler (\emph{Bias}) des Modells zu reduzieren und die Genauigkeit
schrittweise zu verbessern, indem es sich auf schwierig zu klassifizierende Datenpunkte konzentriert.

\begin{figure}[h]
    \centering
    \includegraphics[width=0.8\textwidth]{"./figures/Boosting_Graph"}
    \caption{Flussdiagramm zur Veranschaulichung des Boosting-Konzepts}
\end{figure}

\subsection*{Beispiel:}
Stellen wir uns vor, wir möchten ein Modell entwickeln, das den Preis von
Häusern basierend auf verschiedenen Merkmalen wie Größe, Lage, Anzahl der
Zimmer und Baujahr vorhersagt.

\begin{table}[h]
    \centering
    \begin{tabularx}{\textwidth}{|X|X|X|X|}
    \hline
    \textbf{Haus} & \textbf{Größe $[m^2]$} & \textbf{Lage} & \textbf{Preis} \\
    \hline
    Haus 1        & 100                    & Zentrum       & 500.000€       \\
    \hline
    Haus 2        & 150                    & Vorort        & 300.000€       \\
    \hline
    Haus 3        & 80                     & Zentrum       & 400.000€       \\
    \hline
    Haus 4        & 120                    & Ländlich      & 200.000€       \\
    \hline
\end{tabularx}
    \caption{Beispielhafte Daten für Hauspreise basierend auf Größe und Lage}
\end{table}

\begin{itemize}
    \item Wir entscheiden uns zunächst für ein sehr einfaches Modell (schwacher Lerner),
          das den Preis nur anhand der Größe des Hauses vorhersagt.
          Dieses Modell geht davon aus, dass alle anderen Merkmale keinen
          Einfluss auf den Preis haben.
    \item In Wirklichkeit variieren die Hauspreise jedoch nicht nur aufgrund
          ihrer Größe, sondern auch aufgrund anderer Faktoren. Ein kleines Haus in einer
          begehrten Lage könnte teurer sein als ein großes Haus in einer weniger beliebten
    \item Da unser Modell nur die Größe berücksichtigt und alle anderen Faktoren ignoriert,
          wird es systematisch den Preis von Häusern in begehrten Lagen unterschätzen und den Preis
          von Häusern in weniger beliebten Gegenden überschätzen. Dieser systematische Fehler in den
          Vorhersagen ist der \textbf{Bias}.
\end{itemize}

In mehreren Iterationen wird beim Boosting nun das Gewicht der Datenpunkte so angepasst, dass sich das
zweite Modell auf genau die Datenpunke fokussiert, welche zuvor falsch bzw. besonders schlecht vorhergesagt wurden.
Somit ist hier eine deutlich größere Anpassung der Vorhersage zu erkennen als bei den Datenpunkten, die zuvor
relativ gut vorhergasagt wurden.

\begin{table}
    \centering
    \begin{tabularx}{\textwidth}{|X|X|X|X|X|X|}
    \hline
    \textbf{Haus} & \textbf{Größe $[m^2]$} & \textbf{Lage} & \textbf{Preis} & \textbf{Vorhersage (It. 1)} & \textbf{Vorhersage (It. 2)} \\
    \hline
    Haus 1        & 100                    & Zentrum       & 500.000€       & 450.000€                    & 490.000€                    \\
    \hline
    Haus 2        & 150                    & Vorort        & 300.000€       & 350.000€                    & 310.000€                    \\
    \hline
    Haus 3        & 80                     & Zentrum       & 400.000€       & 380.000€                    & 405.000€                    \\
    \hline
    Haus 4        & 120                    & Ländlich      & 200.000€       & 250.000€                    & 210.000€                    \\
    \hline
\end{tabularx}
    \caption{Beispielhafte Daten für Hauspreise und wie Boosting den Bias in mehreren Iterationen reduziert}
\end{table}
Zuletzt werden alle schwachen Lerner zu einem \emph{starken Lerner} als Ensemble zusammengefügt, wobei die Vorhersagen
der einzelnen Modelle entsprechend ihrer individuellen Präzision gewichtet werden.