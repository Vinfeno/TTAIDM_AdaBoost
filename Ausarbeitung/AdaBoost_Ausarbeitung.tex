\documentclass[11pt,a4paper,oneside]{scrartcl}

\usepackage[utf8]{inputenc}
\usepackage[ngerman]{babel}
\usepackage{amsmath, amssymb, amsthm} 
\usepackage{graphicx, tikz} 
\usepackage{hyperref} 
\usepackage{tabularx}
\usepackage[linesnumbered,ruled,vlined]{algorithm2e}

\newtheorem{satz}{Satz}[section]
\newtheorem{lemma}[satz]{Lemma}
\newtheorem{proposition}[satz]{Proposition}
\newtheorem{definition}[satz]{Definition}
\newtheorem{bemerkung}[satz]{Bemerkung}

\begin{document}

\title{Ausarbeitung zum Thema AdaBoost}
\subtitle{Bachelor-Seminar "`Top 10 Algorithms in Data Mining"'}

\author{Marius Graf}
\date{\today}

\maketitle

\tableofcontents

\begin{abstract}
    \noindent\textbf{Abstract:}
    Diese Arbeit untersucht den AdaBoost-Algorithmus im Kontext des maschinellen Lernens.
    Sie beleuchtet seine Entstehung, Vorteile und Herausforderungen. Anhand von Anwendungsbeispielen aus
    verschiedenen Bereichen wird die Anwendbarkeit von AdaBoost dargestellt. Die Relevanz von AdaBoost in der
    aktuellen Datenwissenschaft wird abschließend erörtert.
\end{abstract}
\newpage

\section{Einleitung}
Data Mining analysiert große Datenmengen, um Muster und Zusammenhänge zu erkennen,
wobei Methoden aus Statistik, Machine Learning und Datenbanktechnologie eingesetzt werden.
Es spielt eine zentrale Rolle in Forschung und Industrie, um Erkenntnisse zu gewinnen und
Entscheidungen zu unterstützen. AdaBoost gehört zu den
\emph{Ensemble-Methoden}, die mehrere Modelle kombinieren, um präzisere Vorhersagen zu treffen.
Diese Arbeit konzentriert sich auf AdaBoost, der Beobachtungen gewichtet, um schwierig zu
klassifizierende Datenpunkte besser vorherzusagen.

\section{Grundlagen des Boosting}
Boosting ist eine Ensemble-Technik im Machine Learning, die mehrere sog. \emph{\glqq schwache Lerner\grqq} kombiniert,
um ein präziseres Gesamtmodell zu erstellen. Als \emph{schwacher Lerner} wird ein Modell bezeichnet, das nur einen geringen
Zusammenhang in den Daten lernen kann und dessen Vorhersagen daher nur etwas besser als zufälliges Raten sind.
In jeder Iteration wird ein solches Modell trainiert, das versucht, falsche
Vorhersagen der vorherigen Modelle zu korrigieren, indem es die Gewichtung der Trainingsdaten anpasst.
Das Ziel ist es, den systematischen Fehler (\emph{Bias}) des Modells zu reduzieren, der durch die einseitige
Betrachtung der Daten entsteht. Dadurch wird in de Regel eine bessere Generalisierbarkeit, also Genauigkeit auf
ungesehenen Daten erhöht.

\begin{figure}[h]
    \centering
    \includegraphics[width=0.8\textwidth]{"./figures/Boosting_Graph"}
    \caption{Flussdiagramm zur Veranschaulichung des Boosting-Konzepts}
\end{figure}

\subsection*{Beispiel:}
(P(Iteration) = Vorhersage, W(Iteration) = Gewichtung)\\
\begin{center}
    \begin{tabularx}{\textwidth}{|X|X|X|X|X|X|}
    \hline
    \textbf{Haus} & \textbf{Größe $[m^2]$} & \textbf{Lage} & \textbf{Preis} & \textbf{Vorhersage (It. 1)} & \textbf{Vorhersage (It. 2)} \\
    \hline
    Haus 1        & 100                    & Zentrum       & 500.000€       & 450.000€                    & 490.000€                    \\
    \hline
    Haus 2        & 150                    & Vorort        & 300.000€       & 350.000€                    & 310.000€                    \\
    \hline
    Haus 3        & 80                     & Zentrum       & 400.000€       & 380.000€                    & 405.000€                    \\
    \hline
    Haus 4        & 120                    & Ländlich      & 200.000€       & 250.000€                    & 210.000€                    \\
    \hline
\end{tabularx}
    \begin{tabularx}{\textwidth}{|X|X|X|X|X|X|}
    \hline
    \textbf{Haus} & \textbf{P(1)} & \textbf{W(1)} & \textbf{P(2)} & \textbf{W(2)} & \textbf{P(3)} \\
    \hline
    Haus 1        & 450.000€      & 0.1307        & 445.000€      & 0.1192        & 443.000€      \\
    \hline
    Haus 2        & 350.000€      & 0.5299        & 495.000€      & 0.4833        & 501.000€      \\
    \hline
    Haus 3        & 380.000€      & 0.1444        & 430.000€      & 0.1691        & 410.000€      \\
    \hline
    Haus 4        & 250.000€      & 0.1950        & 230.000€      & 0.2283        & 205.000€      \\
    \hline
\end{tabularx}\\[5pt]
    \emph{Beispielhafte Anpassung der Gewichte über die Iterationen}
\end{center}


\section{Der AdaBoost Algorithmus}
\subsection*{Einführung}
AdaBoost, kurz für \glqq Adaptive Boosting\grqq, wurde in den 1990ern von Yoav Freund und Robert Schapire für die binäre
Klassifikation konzipiert und hat das Feld des maschinellen Lernens maßgeblich geprägt. Während Boosting-Methoden
allgemein iterativ arbeiten und Fehler der vorherigen Modelle korrigieren, zeichnet sich AdaBoost durch seine spezielle
Methode zur Gewichtungsanpassung der Datenpunkte aus. In jeder Trainingsiteration erhöht AdaBoost gezielt die Gewichtung
der falsch klassifizierten Datenpunkte, wodurch er kontinuierlich seine Vorhersagegenauigkeit verbessert. Dieser
einzigartige und adaptive Ansatz zur Fehlerkorrektur unterscheidet AdaBoost von anderen Boosting-Methoden und hat
nicht nur die Effizienz von Boosting-Methoden für binäre Klassifikationsprobleme unter Beweis gestellt, sondern auch zu
zahlreichen Weiterentwicklungen in diesem Bereich angeregt. \cite{WuKumar2009}

\subsection*{Notation}
\begin{itemize}
    \item Sei $\mathcal{X}$ die Menge der \textbf{Features}.
          $\mathcal{Y}$ die Menge der \textbf{Labels}, die gelernt werden sollen.
          Dabei ist $\mathcal{Y}=\{-1, +1\}$ bei binärer Klassifikation. Ein \textbf{Trainingdatensatz} $D$ besteht aus $m$ Einträgen,
          welche Features mit Labels verbinden:
          $$
              D=\{(\boldsymbol{x}_i, y_i)\},~i=1, \dots, m
          $$
    \item Nach dem Training auf $D$ gibt ein \textbf{Lernalgorithmus} (meistens \emph{Decision Stump}) $\mathcal{L}$ eine \textbf{Hypothese} bzw. einen Klassifizierer
          $h$ zurück, der von $X$ nach $\mathcal{Y}$ abbildet. In der Regel wrid sogar eine Folge von Hypothesen $(h_j)_{j\in\mathcal{I}}$ mit $\mathcal{I}=\{1,..,n\}$ zurückgegeben.
          \begin{align*}
              h_j:X \rightarrow \mathcal{Y},~h_j(\boldsymbol{x}) = y
          \end{align*}
    \item $T$ ist die Anzahl der gewünschten \textbf{Trainingsiterationen}.
    \item Bei jeder Iteration $t=1, \dots,T$ wird der Datensatz um \textbf{Gewichte} $$
              w_i^{(t)},~i=1,..,m
          $$ erweitert.
\end{itemize}


\subsection*{Initialisierung der Gewichte}
Zu Beginn sind die Gewichte aller $m$ Datenpunkte gleich verteilt:
$$
    \mathcal{D}_1(i) = \frac{1}{m}
$$
Zudem ist die Summe aller Gewichte bei jeder Iteration stets $1$.
$$
    \sum_{i=1}^n \mathcal{D}_t(i) = 1
$$

\subsection*{Training der schwachen Lerner}
Trainiere für $t=1,\dots,T$ Iterationen mehrere schwache Lerner unter berücksichtigung der aktuellen Gewichtung.
$$
    h = \mathcal{L}(D, w_t)
$$
Ziel ist es, den besten Lerner in der Iteration auszuwählen und diesen dem Ensemble hinzuzufügen. Dazu werden
die Gewichte der falsch vorhergesagten Datenpunkte summiert:
$$
    \varepsilon_j = \sum_{i=1}^m w^{(t)}_i\cdot I\left(y_i \neq h_j\left(\boldsymbol{x}_i\right)\right)
$$
$I$ bezeichnet die Indikatorfunktion, die $1$ zurückgibt, wenn $y_i\neq h_j(\boldsymbol{x_i})$ erfüllt ist, und
sonst $0$.
Das Modell mit dem geringsten Fehler $\varepsilon_j$ wird in der Iteration als $h_t$ mit Fehler $\varepsilon_t$ ausgewählt.

\subsection*{Berechnung des Lernkoeffizienten}
Der Koeffizient $\alpha_t$ für den ausgewählen schwachen Lerner $h_t$ wird wie folgt berechnet:
$$
    \alpha_t = \frac{1}{2}\ln\left(\frac{1-\varepsilon_t}{\varepsilon_t}\right)
$$
Dieser gibt an, wie stark die Vorhersage dieses schwachen Lerners im späteren Ensemble
gewichtet wird.

\subsection*{Aktualisierung der Gewichte}
Die Gewichte der Trainingsdaten werden basierend auf
dem zuvor bestimmten Koeffizienten wie folgt aktualisiert:
Die Gewichte der Trainingsdaten werden basierend auf
dem zuvor bestimmten Koeffizienten wie folgt aktualisiert:
\input{module/AdaBoost/formeln/Aktualisierung.tex}
Dabei werden alle neuen Gewichte zuletzt normalisiert,
damit ihre Summe nach der Aktualisierung wieder $1$ ist.
\input{module/AdaBoost/formeln/Normalisierung.tex}
Dabei werden alle neuen Gewichte zuletzt normalisiert,
damit ihre Summe nach der Aktualisierung wieder $1$ ist.
\begin{align*}
    Z_t         & =\sum_{j=1}^m w^{(t+1)}_i\qquad(\text{Normalisierungsfaktor}) \\
    w^{(t+1)}_i & = \frac{w^{(t+1)}_i}{Z_t}
\end{align*}

\subsection*{Das Ergebnis des Algorithmus}
Der Algorithmus gibt ein Gesamtmodell zurück, welches die Klassifizierung des Datenpunktes durch die gewichtete
Summe aller schwachen Lerner darstellt:
\begin{align*}
    H    & :      X \rightarrow \{-1, +1\}                       \\
    H(x) & =  \text{sign}\left(\sum_{t=1}^T\alpha_th_t(x)\right)
\end{align*}
\begin{algorithm}[H]
    \DontPrintSemicolon
    \LinesNotNumbered
    \KwData{Trainingsdatensatz \(D\), Anzahl der Iterationen \(T\).}
    \KwResult{Finale Klassifikationsfunktion: \(H(x) = \text{sign}\left(\sum_{t=1}^{T} \alpha_t h_t(x)\right)\).}
    \BlankLine
    \tcp{Initialisiere Gewichte}
    \(w^{(1)}_i=\frac{1}{m}\)\;
    \For{\(t = 1\) \KwTo \(T\)}{
    \tcp{Trainiere schwache Lerner}
    \((h_{j})_{j\in\mathcal{I}} \leftarrow \mathcal{L}(D, w^{(t)}_i)\) \;
    \tcp{Berechne Fehler}
    \For{$j=1$ \KwTo $n$ }{
        \(\varepsilon_j = \sum_{i=1}^{m} w^{(t)}_i\cdot I(y_i \neq h_j(x_i))\)\;
    }
    Wähle Lerner $h_j$ mit minimalem Fehler $\varepsilon_j$ als \(h_t\) mit Fehler $\varepsilon_t$\;
    \tcp{Berechne den Lernerkoeffizienten}
    \(\alpha_t = \frac{1}{2} \ln \left( \frac{1 - \varepsilon_t}{\varepsilon_t} \right)\)\;
    \tcp{Aktualisiere die Gewichte für die nächsten Iterationen}
    \eIf{\(y_i = h_t(x_i)\)}{
        \(w^{(t+1)}_i \leftarrow w^{(t)}_i \cdot e^{-\alpha_t}\)\;
    }{
        \(w^{(t+1)}_i \leftarrow w^{(t)}_i \cdot e^{\alpha_t}\)\;
    }
    \tcp{Normalisiere Gewichte}
    \(Z_t \leftarrow \sum_{j=1}^mw^{(t+1)}_i\)\;
    \For{\(i = 1\) \KwTo \(m\)}{
        \(w^{(t+1)}_i \leftarrow \frac{w^{(t+1)}_i}{Z_t}\)\;
    }
    }
    \KwOut{\(H(x)=\text{sign}\left(\sum_{t=1}^T\alpha_th_t(x)\right)\)}
    \tcp{Ende des Algorithmus}
\end{algorithm}



% \input{../Ausarbeitung/module/AdaBoost/formeln/Algo1.tex}
% \input{../Ausarbeitung/module/AdaBoost/formeln/Algo2.tex}

\newpage
\subsection*{Illustration des Algorithmus: das XOR-Problem}
(Entnommen aus \glqq The Top 10 Algorithms in Data Mining\grqq~\cite{WuKumar2009}) \\[10pt]
Ein Datensatz besteht aus nur 4 Datenpunken:
$$
    \left\{
    \begin{array}{c}
        (x_1 =(0, +1), y_1=+1) \\
        (x_2 =(0, -1), y_2=+1) \\
        (x_3 =(+1, 0), y_3=-1) \\
        (x_4 =(-1, 0), y_4=-1)
    \end{array}
    \right\}
$$
\begin{figure*}
    \centering
    \includegraphics[width=0.5\textwidth]{figures/XOR-Problem.png}
    \caption[]{Visualisierung des XOR-Problems}
    \label{fig:XOR-Problem}
\end{figure*}
Wie aus Abbildung \ref*{fig:XOR-Problem} leicht zu erkennen ist, kann keine einfache Trennlinie gezogen werden, um
$+1$ und $-1$ voneinander zu trennen.\\
Nehmen wir nun an, dass durch den Lernalgorithmus nun acht Modelle als Funktionen vorliegen:
\begin{align*}
    h_1(x)=\left\{\begin{array}{r c}
                      +1, & \text{ wenn } (x_1 > -0.5) \\
                      -1, & \text{sonst}
                  \end{array}\right. &
    h_2(x)=\left\{\begin{array}{r c}
                      -1, & \text{ wenn } (x_1 > -0.5) \\
                      +1, & \text{sonst}
                  \end{array}\right. \\[10pt]
    h_3(x)=\left\{\begin{array}{r c}
                      +1, & \text{ wenn } (x_1 > +0.5) \\
                      -1, & \text{sonst}
                  \end{array}\right. &
    h_4(x)=\left\{\begin{array}{r c}
                      -1, & \text{ wenn } (x_1 > +0.5) \\
                      +1, & \text{sonst}
                  \end{array}\right. \\[10pt]
    h_5(x)=\left\{\begin{array}{r c}
                      +1, & \text{ wenn } (x_2 > -0.5) \\
                      -1, & \text{sonst}
                  \end{array}\right. &
    h_6(x)=\left\{\begin{array}{r c}
                      -1, & \text{ wenn } (x_2 > -0.5) \\
                      +1, & \text{sonst}
                  \end{array}\right. \\[10pt]
    h_7(x)=\left\{\begin{array}{r c}
                      +1, & \text{ wenn } (x_2 > +0.5) \\
                      -1, & \text{sonst}
                  \end{array}\right. &
    h_8(x)=\left\{\begin{array}{r c}
                      -1, & \text{ wenn } (x_2 > +0.5) \\
                      +1, & \text{sonst}
                  \end{array}\right. \\[10pt]
\end{align*}
\begin{enumerate}
    \item Der erste Schritt besteht darin, den Basis-Lernalgorithmus auf den ursprünglichen Daten aufzurufen.
          $h_2, h_3, h_5$ und $h_8$ haben alle eine Klassifikationsfehler von 0.25. Angenommen, $h_2$ wird als erster
          Basis-Lerner ausgewählt. Ein Datensatz $(x_1)$ wird falsch klassifiziert, daher beträgt der Fehler 0.25.
          Das Gewicht von $h_2$ beträgt ungefähr $\varepsilon_t\approx 0.55$. Abbildung \ref*{fig:XOR-Solution}(b) zeigt die Klassifikation und die Gewichtungen.
    \item Das Gewicht von $x_1$ wird erhöht und der Basis-Lernalgorithmus erneut aufgerufen. Diesmal haben $h_3, h_5$ und $h_8$
          gleiche Fehler. Angenommen, $h_3$ wird ausgewählt, dessen Gewicht 0.80 beträgt. Abbildung \ref*{fig:XOR-Solution}(c)
          zeigt die kombinierte Klassifikation von $h_2$ und $h_3$.
    \item Das Gewicht von $x_3$ wird erhöht. Diesmal haben nur $h_5$ und $h_8$ die niedrigsten Fehler. Angenommen, $h_5$ wird
          ausgewählt, dessen Gewicht 1.10 beträgt. Abbildung \ref*{fig:XOR-Solution}(d) zeigt die kombinierte Klassifikation
          von $h_2, h_3$ und $h_8$. Durch die Kombination der unvollkommenen linearen Klassifikatoren hat AdaBoost einen nichtlinearen Klassifikator mit
          null Fehler erzeugt.
\end{enumerate}
\textbf{Siehe auch entsprechende Implementierung: \cite{graf2023repository}}
\begin{figure*}
    \centering
    \includegraphics[width=0.7\textwidth]{figures/XOR_Solution.png}
    \caption[]{Visualisierung von AdaBoost auf dem XOR-Problem}
    \label{fig:XOR-Solution}
\end{figure*}

\section{Praktische Anwendung und Beispiele}
Vor allem wegen seiner Effizienz und Genauigkeit hat sich  in einer Vielzahl
von Anwendungen bewährt:
\begin{itemize}
    \item \textbf{Bilderkennung und Computervision:} Gesichtserkennung~\cite{viola2001rapid}
    \item \textbf{Textklassifikation und Natural Language Processing}: Erkennung von Spam-Mail~\cite{panwar2022detection}
    \item \textbf{Medizinische Diagnostik:} Risiko/Erkennung von Krankheiten baserend auf Patientendaten~\cite{hatwell2020ada}
    \item \textbf{Finanzwesen:} Vorhersage von Aktienkursbewegungen~\cite{zhang2016stock}
\end{itemize}
\begin{figure*}
    \centering
    \includegraphics[width=.65\textwidth]{figures/CV_Example.png}
    \caption{Anwendung von AdaBoost bei Computer Vision:
        Das erste Merkmal von AdaBoost misst den Intensitätsunterschied
        zwischen der Augenregion und den oberen Wangen,
        wobei die Augen oft dunkler sind. Das zweite Merkmal vergleicht die
        Intensität der Augen mit der Nasenbrücke.\cite{viola2001rapid}}
\end{figure*}

\section{Vor- und Nachteile von AdaBoost}
AdaBoost ist einfach zu implementieren, vielseitig und benötigt oft keine Anpassung des Basislerners.
Er ist weniger anfällig für Overfitting und kann wichtige Features automatisch identifizieren.\\\\
Nachteile sind seine Empfindlichkeit gegenüber verrauschten Daten und Außreißern, der potenzielle Zeitaufwand bei großen
Datensätzen und seine Abhängigkeit vom Basislerner sowie seine Ausrichtung auf binäre Klassifikation.

\section{Erweiterungen und Variationen von AdaBoost}
AdaBoost, ursprünglich für binäre Klassifikation entwickelt,
wurde durch verschiedene Erweiterungen für diverse Problemstellungen adaptiert.

\begin{itemize}
    \item Variationen wie \glqq AdaBoost.M1\grqq~ und \glqq SAMME\grqq~ erweitern den Algorithmus für Multiklassen-Probleme. \cite{hastie2009multi}
    \item Kosten-sensitives AdaBoost passt Gewichtungen basierend auf bestimmten Fehlerarten an. \cite{masnadi2010cost}
    \item Neben Entscheidungsstümpfen kann AdaBoost mit SVMs, Neuronalen Netzen oder jedem anderen Klassifikator kombiniert werden. \cite{zhang2016stock}
    \item Robuste AdaBoost-Varianten minimieren die Auswirkung von Ausreißern. \cite{viola2001fast}
    \item Online AdaBoost aktualisiert Modelle mit sequenziellen Daten ohne Neutrainierung. \cite{hu2013online}
    \item Einige Varianten integrieren Feature-Auswahl direkt, um Interpretierbarkeit und Trainingseffizienz zu steigern. \cite{wu2003learning}
\end{itemize}


\section{Schlusswort}
AdaBoost hat im Bereich des Ensemble-Lernens das maschinelle Lernen beeinflusst. Durch seine Einfachheit
und Leistungsfähigkeit ist er ein bis heute relevantes Werkzeug für Datenwissenschaftler.

\bibliographystyle{alpha}
\bibliography{seminar_top10.bib}

\end{document}