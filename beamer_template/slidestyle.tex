\usetheme{HPC}

% Set language options
\if\englishlanguage0
\PassOptionsToPackage{german,germankw,onelanguage}{algorithm2e}
\PassOptionsToPackage{ngerman}{babel}
\else
\PassOptionsToPackage{english}{babel}
\fi

%%%%%%%%%%%%%%%%%%%%%%%%%%%%%%%%%%%%%%%%%%%%%%%%%%%%%%%%%%%%%%%%%%%%%%%%%
% Include useful packages 
\usepackage[utf8]{inputenc}
\usepackage{amsmath}
\usepackage{framed}
\usepackage[ruled,vlined]{algorithm2e}
\usepackage{amssymb}
\usepackage{array}
\usepackage{caption}
\usepackage{bbding}
\usepackage{bm}
\usepackage{hyperref}
\usepackage{tikz}
\usepackage{times}
\usepackage{ifthen}
\usepackage{pgfplots}
\usepackage{alltt}
\usepackage{transparent}
\usepackage{colortbl}
\usepackage{textcomp}
\usepackage{multirow}
\usepackage{babel}


% increase itemize spacing
\let\realitemize\itemize
\let\endrealitemize\enditemize
\renewenvironment{itemize}{%
    \realitemize\setlength{\parskip}{0pt}\setlength{\itemsep}{.24cm}}
    {%
    \endrealitemize%
    }

% Switch between transparency and invisibility for covered things
\if\coveredtransparent1
\setbeamercovered{transparent}
\fi

%%%%%%%%%%%%%%%%%%%%%%%%%%%%%%%%%%%%%%%%%%%%%%%%%%%%%%%%%%%%%%%%%%%%%%%%%
% TikZ initializations, in particular for overlays and externalization

\pgfplotsset{compat=1.12}

\tikzstyle{every picture}+=[remember picture]
\tikzstyle{na} = [baseline=-.5ex, xshift = -0.15cm]
\tikzstyle{na2} = [baseline=-.5ex, xshift = -0.35cm]

\tikzset{
    ncbar angle/.initial=90,
    ncbar/.style={
        to path=(\tikztostart)
        -- ($(\tikztostart)!#1!\pgfkeysvalueof{/tikz/ncbar angle}:(\tikztotarget)$)
        -- ($(\tikztotarget)!($(\tikztostart)!#1!\pgfkeysvalueof{/tikz/ncbar angle}:(\tikztotarget)$)!\pgfkeysvalueof{/tikz/ncbar angle}:(\tikztostart)$)
        -- (\tikztotarget)
    },
    ncbar/.default=0.5cm,
}

\tikzset{square left brace/.style={ncbar=0.5cm}}
\tikzset{square right brace/.style={ncbar=-0.5cm}}

\tikzset{round left paren/.style={ncbar=0.5cm,out=120,in=-120}}
\tikzset{round right paren/.style={ncbar=0.5cm,out=60,in=-60}}
\usetikzlibrary{positioning,matrix,arrows,arrows.meta,backgrounds,shapes}
\usetikzlibrary{backgrounds,mindmap,decorations.pathreplacing,external,calc}
\tikzexternalize[prefix=tikzfigures/] % path for saving precompiled tikz pictures

% Command for easily connecting tikz anchors on slide by arrow
\newcommand{\connectbyarrow}[3]{%
\begin{tikzpicture}[overlay]
        \path[->,UniGruen,very thick,shorten >= .25cm, shorten <= .25cm] (#1) edge [#3] (#2);
\end{tikzpicture}
}

%%%%%%%%%%%%%%%%%%%%%%%%%%%%%%%%%%%%%%%%%%%%%%%%%%%%%%%%%%%%%%%%%%%%%%%%%
% Some layout stuff

% set custom text margins
\setbeamersize{text margin left=1em,text margin right=1em}

% Turn off navigation symbols if desired
\if\printnavigationsymbols0
\beamertemplatenavigationsymbolsempty
\fi

% footline design
\setbeamertemplate{footline}{
  \begin{beamercolorbox}[sep=2pt]{footline}
    \hspace{1em}\insertshortauthor\\ \textit{\insertshorttitle{} \ifthenelse{\printcoursename=1}{(\coursenamefootline)}{}} \hfill
    \ifthenelse{\printpagenumbers=1}{\insertframenumber/\inserttotalframenumber}{} \hspace{1pt}
\end{beamercolorbox}  
}

% custom commands for removing some slides from miniframes (needed for table of contents
% at beginning of section/subsection, see below)
\makeatletter
    \let\beamer@writeslidentry@miniframeson=\beamer@writeslidentry
    \def\beamer@writeslidentry@miniframesoff{%
      \expandafter\beamer@ifempty\expandafter{\beamer@framestartpage}{}%
      {%
        \clearpage\beamer@notesactions%
      }
    }
    \newcommand*{\miniframeson}{\let\beamer@writeslidentry=\beamer@writeslidentry@miniframeson}
    \newcommand*{\miniframesoff}{\let\beamer@writeslidentry=\beamer@writeslidentry@miniframesoff}
    \beamer@compresstrue
\makeatother

% Print table of contents at beginning of each section, with the current section highlighted in green and everything else shaded
\if\printtocatbeginofsection1
\AtBeginSection[]
{   
		% Remove frame number from footline on outline slides
		\let\rememberpagenumberswitch\printpagenumbers
		\def\printpagenumbers{0}
		\miniframesoff
	  \begin{frame}[t,noframenumbering]{\ifthenelse{\englishlanguage=1}{Outline}{Inhalt}}
			\setbeamercolor{section in toc}{fg=UniGruen,bg=}
			\setbeamercolor{section in toc shaded}{fg=Gray,bg=}
			\setbeamercolor{subsection in toc shaded}{fg=Gray,bg=}
			\setbeamercolor{subsection in toc}{fg=Gray,bg=}
			\tableofcontents[currentsection]
    \end{frame}
		\miniframeson
		\let\printpagenumbers\rememberpagenumberswitch
}
\fi

% Print table of contents at beginning of each subsection, with the current section and subsection highlighted in green and everything else shaded
\if\printtocatbeginofsubsection1
\AtBeginSubsection[]
{
			% Remove frame number from footline on outline slides
			\let\rememberpagenumberswitch\printpagenumbers
			\def\printpagenumbers{0}
		  \miniframesoff
      \begin{frame}[t,noframenumbering]{\ifthenelse{\englishlanguage=1}{Outline}{Inhalt}}
			\setbeamercolor{section in toc}{fg=UniGruen,bg=}
			\setbeamercolor{section in toc shaded}{fg=Gray,bg=}
			\setbeamercolor{subsection in toc shaded}{fg=Gray,bg=}
			\setbeamercolor{subsection in toc}{fg=UniGruen,bg=}
			\tableofcontents[currentsection,currentsubsection]
    \end{frame}
		\miniframeson
		\let\printpagenumbers\rememberpagenumberswitch
}
\fi

% Include small "Uni-Loewe" icon in top right corner of each slide
\if\printlion1
\addtobeamertemplate{frametitle}{}{%
\tikzexternaldisable%
\begin{tikzpicture}[remember picture,overlay]%
\node[anchor=north east,yshift=1.5pt, xshift = 0.2pt] at (current page.north east) {\includegraphics[height=.6cm]{figures/loewe-weiss.pdf}};%
\end{tikzpicture}%
\tikzexternalenable%

\vspace{-.5cm}
}
\fi

%%%%%%%%%%%%%%%%%%%%%%%%%%%%%%%%%%%%%%%%%%%%%%%%%%%%%%%%%%%%%%%%%%%%%%%%%
% Title page

% Do not count title page in page numbering
\let\otp\titlepage
\renewcommand{\titlepage}{\otp\addtocounter{framenumber}{-1}}

% Custom title page layout
\defbeamertemplate*{title page}{customized}
{
\thispagestyle{empty}

%\vspace{-.5cm}
\if\printbanner1
\hpcbanner
\fi

	\vspace*{.25cm}
	\begin{center}
	\if\printcoursename1
	\Large\textbf{\coursename}\par
	\fi
	\bigskip
	\hfill
	\begin{beamercolorbox}[rounded=true, center, wd=.8\paperwidth]{mycolor}
  \Large\inserttitle
  \end{beamercolorbox}
  \hfill\hfill
	
  \bigskip
	\bigskip
  
	\if\printauthor1
	\normalsize\textbf{\insertauthor}\par
	\fi
	
	\vspace{.15cm}
	\bigskip
  \if\printinstitute1
	\small\insertinstitute\par
	\fi
	\bigskip
	\if\printdate1
  \normalsize\insertdate\par
	\fi
	\ifx\longtitle\undefined
	\else
		\if\longtitle1
		\vspace{-1cm}
		\else
		
		\fi
	\fi
	\end{center}
}

% Generate HPC banner for title page, similar to our exercise sheet headers etc.
\newcommand{\hpcbanner}{
	\makebox[\textwidth][c]{
		\begin{tikzpicture}
			[textnode/.style={white,font={\bf \sffamily \small},inner sep=0pt}]
						\fill [UniGruen] (0,0) rectangle (\paperwidth,2cm);
			\node [inner sep=0pt] (loewe) at (1,1) {\includegraphics[width=1.75cm]{figures/loewe-weiss.pdf}};
			\node [textnode,anchor=west] (T) at (2.25,1) {\ifthenelse{\englishlanguage=1}{Scientific Computing \& High Performance Computing}{Wissenschaftliches Rechnen und Hochleistungsrechnen}};
			\node [textnode,above=0.6cm of T.west,anchor=west] {Bergische Universität Wuppertal};
			\node [textnode,below=0.6cm of T.west,anchor=west] {\lecturername};
		\end{tikzpicture}
	}
}

%%%%%%%%%%%%%%%%%%%%%%%%%%%%%%%%%%%%%%%%%%%%%%%%%%%%%%%%%%%%%%%%%%%%%%%%%
% Auxiliary stuff

% Remove algorithm numbering
\renewcommand{\thealgocf}{}

% Command for including small book icon that can be used for referencing literature, lecture notes or similar
\newcommand{\smallbook}{\includegraphics[width = .028\textwidth]{figures/bookicon}}

% Command that generates a framed box containing a book icon and text
\newcommand{\inbook}[1]{{\setlength{\topsep}{0pt}
\begin{framed}
\begin{minipage}{.08\textwidth}
\includegraphics[width = .99\textwidth]{figures/bookicon}
\end{minipage}
\begin{minipage}{.9\textwidth}
#1
\end{minipage}
\end{framed}}}

% Slight change of bibliography layout to look better on slides
\let\OLDthebibliography\thebibliography
\renewcommand\thebibliography[1]{
  \OLDthebibliography{#1}
  \setlength{\parskip}{0pt}
  \setlength{\itemsep}{0pt plus 0.3ex}
}

%%%%%%%%%%%%%%%%%%%%%%%%%%%%%%%%%%%%%%%%%%%%%%%%%%%%%%%%%%%%%%%%%%%%%%%%%
% Some commands that I frequently need
\newcolumntype{C}[1]{>{\centering\arraybackslash}p{#1}}
% Define your logos here to be used on the title page
\newcommand{\BUWLogo}{\includegraphics[height=32pt]{figures/buw}}
\newcommand{\upk}{^{(k)}}
\newcommand{\upm}{^{(m)}}
\newcommand{\upinv}{^{-1}}
\newcommand{\R}{\mathbb{R}}
\newcommand{\N}{\mathbb{N}}
\newcommand{\C}{\mathbb{C}}
\newcommand{\K}{\mathcal{K}}
\newcommand{\bigO}{\mathcal{O}}
\newcommand{\Rn}{\mathbb{R}^n}
\newcommand{\Rk}{\mathbb{R}^k}
\newcommand{\Cn}{\mathbb{C}^n}
\newcommand{\Rnn}{\mathbb{R}^{n \times n}}
\newcommand{\Cnn}{\mathbb{C}^{n \times n}}
\newcommand{\slitplane}{\mathbb{C} \setminus \mathbb{R}_0^-}
\newcommand{\va}{{\mathbf a}}
\newcommand{\vb}{{\mathbf b}}
\newcommand{\vc}{{\mathbf c}}
\newcommand{\vd}{{\mathbf d}}
\newcommand{\vdhat}{{\mathbf {\hat d}}}
\newcommand{\ve}{{\mathbf e}}
\newcommand{\vehat}{{\mathbf {\hat e}}}
\newcommand{\vf}{{\mathbf f}}
\newcommand{\vftilde}{{\mathbf {\widetilde f}}}
\newcommand{\vg}{{\mathbf g}}
\newcommand{\vh}{{\mathbf h}}
\newcommand{\vhhat}{{\mathbf {\hat h}}}
\newcommand{\vk}{{\mathbf k}}
\newcommand{\vp}{{\mathbf p}}
\newcommand{\vq}{{\mathbf q}}
\newcommand{\vr}{{\mathbf r}}
\newcommand{\vs}{{\mathbf s}}
\newcommand{\vt}{{\mathbf t}}
\newcommand{\vu}{{\mathbf u}}
\newcommand{\vv}{{\mathbf v}}
\newcommand{\vw}{{\mathbf w}}
\newcommand{\vx}{{\mathbf x}}
\newcommand{\vy}{{\mathbf y}}
\newcommand{\vz}{{\mathbf z}}
\newcommand{\vnull}{\boldsymbol{0}}
\newcommand{\vone}{\boldsymbol{1}}
\newcommand{\spK}{{\cal K}}
\newcommand{\spEK}{{\cal E}}
\newcommand{\spL}{{\cal L}}
\newcommand{\spP}{{\cal P}}
\newcommand{\tol}{\texttt{tol}{}}
\newcommand{\specialcell}[2][l]{\begin{tabular}[#1]{@{}c@{}}#2\end{tabular}}
\renewcommand{\d}{\,\mathrm{d}}
\newcommand{\dmu}{\d\mu(t)}
\newcommand{\dalpha}{\d\alpha(z)}
\DeclareMathOperator{\Pe}{Pe}
\DeclareMathOperator{\spec}{spec}
\DeclareMathOperator{\diag}{diag}
\DeclareMathOperator{\range}{range}
\DeclareMathOperator{\Span}{span}
\DeclareMathOperator{\trace}{trace}
\DeclareMathOperator{\tr}{tr}
\DeclareMathOperator{\sign}{sign}
\DeclareMathOperator*{\argmin}{arg\,min}
\DeclareMathOperator*{\argmax}{arg\,max}
\newcommand{\lmin}{{\lambda_{\min}}}
\newcommand{\lmax}{{\lambda_{\max}}}
\newcommand{\AHA}{A^H\!A}
\newcommand{\tve}{\widetilde{{\mathbf e}}}
\newcommand{\tvf}{\widetilde{{\mathbf f}}}
\newcommand{\tvx}{\widetilde{{\mathbf x}}}
\newcommand{\tvr}{\widetilde{{\mathbf r}}}
\newcommand{\rhoinvA}{\rho}
\newcommand{\deltaA}{\delta}
\newcommand{\deltainvA}{\delta'}
\newcommand{\Lmax}{\Lambda_{\max}}
\newcommand{\nmin}{\nu_{\min}}
\newcommand{\nmax}{\nu_{\max}}
\newcommand{\calO}{\mathcal{O}}
\newcommand{\comment}[1]{{\small\color{gray!50}// #1}}
\DeclareMathAlphabet{\mathbf}{OT1}{cmr}{bx}{n}
\def\Hat{\mkern-3mu\text{\textasciicircum}}
