%%%%%%%%%%%%%%%%%%%%%%%%%%%%%%%%%%%%%%%%%%%%%%%%%%%%%%%%%%%%%%%%%%%%%%%%%
%
% Presentation template for Scientific Computing seminar
%
%%%%%%%%%%%%%%%%%%%%%%%%%%%%%%%%%%%%%%%%%%%%%%%%%%%%%%%%%%%%%%%%%%%%%%%%%

\documentclass[hyperref={bookmarks=false},11pt,dvipsnames]{beamer}

%%%%%%%%%%%%%%%%%%%%%%%%%%%%%%%%%%%%%%%%%%%%%%%%%%%%%%%%%%%%%%%%%%%%%%%%%
% Course metadata
\newcommand{\coursename}{Bachelor-Seminar \glqq{}Top 10 Algorithms in Data Mining\grqq{}}
\newcommand{\coursenamefootline}{Bachelor-Seminar \glqq{}Top 10 Algorithms in Data Mining\grqq{}}

% In seminars, the presenter of a talk is typically different from the lecturer responsible for the course.
% The following commands allow to have this distinction. The lecturer's name is printed in the banner on the 
% titlepage (if activated) and the presenter's name is passed into the \author{} command.
\newcommand{\presentername}{Max Mustermann}
\newcommand{\presenternameshort}{M.~Mustermann}
\newcommand{\lecturername}{Dr.~Marcel Schweitzer}

\author[\presenternameshort]{\presentername}
\institute{Bergische Universität Wuppertal}
\def\englishlanguage{0}             % set to 1 to switch from German to English

%%%%%%%%%%%%%%%%%%%%%%%%%%%%%%%%%%%%%%%%%%%%%%%%%%%%%%%%%%%%%%%%%%%%%%%%%
% Switches for title page design
\def\printbanner{1}                 % set to 1 to print title page banner
\def\printauthor{1}                 % set to 1 if author name should also be printed outside of title banner
\def\printdate{1}                   % set to 1 in order to print date
\def\printinstitute{0} 		          % set to 1 in order to print institute
\def\printcoursename{1}		          % set to 1 to print course name above title and in footline

%%%%%%%%%%%%%%%%%%%%%%%%%%%%%%%%%%%%%%%%%%%%%%%%%%%%%%%%%%%%%%%%%%%%%%%%%
% Other layout switches
\def\coveredtransparent{1}          % set to 1 to make covered items completely invisible
\def\printnavigationsymbols{0}      % set to 1 to activate navigation symbols
\def\printtocatbeginofsection{1}    % print outline slide (with highlighted current section) at beginning of each section
\def\printtocatbeginofsubsection{0} % print outline slide (with highlighted current subsection) at beginning of each subsection
\def\printlion{1}                   % set to 0 to suppress small "Uni-Loewe" icon in top right corner
\def\printpagenumbers{1}            % set to 0 to suppress page numbers in foot line
\def\longtitle{0}                   % Sometimes, very long titles can break the title page layout. In that case, setting this to 1 might improve things

%%%%%%%%%%%%%%%%%%%%%%%%%%%%%%%%%%%%%%%%%%%%%%%%%%%%%%%%%%%%%%%%%%%%%%%%%
\usetheme{HPC}

% Set language options
\if\englishlanguage0
\PassOptionsToPackage{german,germankw,onelanguage}{algorithm2e}
\PassOptionsToPackage{ngerman}{babel}
\else
\PassOptionsToPackage{english}{babel}
\fi

%%%%%%%%%%%%%%%%%%%%%%%%%%%%%%%%%%%%%%%%%%%%%%%%%%%%%%%%%%%%%%%%%%%%%%%%%
% Include useful packages 
\usepackage[utf8]{inputenc}
\usepackage{amsmath}
\usepackage{framed}
\usepackage[ruled,vlined]{algorithm2e}
\usepackage{amssymb}
\usepackage{array}
\usepackage{caption}
\usepackage{bbding}
\usepackage{bm}
\usepackage{hyperref}
\usepackage{tikz}
\usepackage{times}
\usepackage{ifthen}
\usepackage{pgfplots}
\usepackage{alltt}
\usepackage{transparent}
\usepackage{colortbl}
\usepackage{textcomp}
\usepackage{multirow}
\usepackage{babel}


% increase itemize spacing
\let\realitemize\itemize
\let\endrealitemize\enditemize
\renewenvironment{itemize}{%
    \realitemize\setlength{\parskip}{0pt}\setlength{\itemsep}{.24cm}}
    {%
    \endrealitemize%
    }

% Switch between transparency and invisibility for covered things
\if\coveredtransparent1
\setbeamercovered{transparent}
\fi

%%%%%%%%%%%%%%%%%%%%%%%%%%%%%%%%%%%%%%%%%%%%%%%%%%%%%%%%%%%%%%%%%%%%%%%%%
% TikZ initializations, in particular for overlays and externalization

\pgfplotsset{compat=1.12}

\tikzstyle{every picture}+=[remember picture]
\tikzstyle{na} = [baseline=-.5ex, xshift = -0.15cm]
\tikzstyle{na2} = [baseline=-.5ex, xshift = -0.35cm]

\tikzset{
    ncbar angle/.initial=90,
    ncbar/.style={
        to path=(\tikztostart)
        -- ($(\tikztostart)!#1!\pgfkeysvalueof{/tikz/ncbar angle}:(\tikztotarget)$)
        -- ($(\tikztotarget)!($(\tikztostart)!#1!\pgfkeysvalueof{/tikz/ncbar angle}:(\tikztotarget)$)!\pgfkeysvalueof{/tikz/ncbar angle}:(\tikztostart)$)
        -- (\tikztotarget)
    },
    ncbar/.default=0.5cm,
}

\tikzset{square left brace/.style={ncbar=0.5cm}}
\tikzset{square right brace/.style={ncbar=-0.5cm}}

\tikzset{round left paren/.style={ncbar=0.5cm,out=120,in=-120}}
\tikzset{round right paren/.style={ncbar=0.5cm,out=60,in=-60}}
\usetikzlibrary{positioning,matrix,arrows,arrows.meta,backgrounds,shapes}
\usetikzlibrary{backgrounds,mindmap,decorations.pathreplacing,external,calc}
\tikzexternalize[prefix=tikzfigures/] % path for saving precompiled tikz pictures

% Command for easily connecting tikz anchors on slide by arrow
\newcommand{\connectbyarrow}[3]{%
\begin{tikzpicture}[overlay]
        \path[->,UniGruen,very thick,shorten >= .25cm, shorten <= .25cm] (#1) edge [#3] (#2);
\end{tikzpicture}
}

%%%%%%%%%%%%%%%%%%%%%%%%%%%%%%%%%%%%%%%%%%%%%%%%%%%%%%%%%%%%%%%%%%%%%%%%%
% Some layout stuff

% set custom text margins
\setbeamersize{text margin left=1em,text margin right=1em}

% Turn off navigation symbols if desired
\if\printnavigationsymbols0
\beamertemplatenavigationsymbolsempty
\fi

% footline design
\setbeamertemplate{footline}{
  \begin{beamercolorbox}[sep=2pt]{footline}
    \hspace{1em}\insertshortauthor\\ \textit{\insertshorttitle{} \ifthenelse{\printcoursename=1}{(\coursenamefootline)}{}} \hfill
    \ifthenelse{\printpagenumbers=1}{\insertframenumber/\inserttotalframenumber}{} \hspace{1pt}
\end{beamercolorbox}  
}

% custom commands for removing some slides from miniframes (needed for table of contents
% at beginning of section/subsection, see below)
\makeatletter
    \let\beamer@writeslidentry@miniframeson=\beamer@writeslidentry
    \def\beamer@writeslidentry@miniframesoff{%
      \expandafter\beamer@ifempty\expandafter{\beamer@framestartpage}{}%
      {%
        \clearpage\beamer@notesactions%
      }
    }
    \newcommand*{\miniframeson}{\let\beamer@writeslidentry=\beamer@writeslidentry@miniframeson}
    \newcommand*{\miniframesoff}{\let\beamer@writeslidentry=\beamer@writeslidentry@miniframesoff}
    \beamer@compresstrue
\makeatother

% Print table of contents at beginning of each section, with the current section highlighted in green and everything else shaded
\if\printtocatbeginofsection1
\AtBeginSection[]
{   
		% Remove frame number from footline on outline slides
		\let\rememberpagenumberswitch\printpagenumbers
		\def\printpagenumbers{0}
		\miniframesoff
	  \begin{frame}[t,noframenumbering]{\ifthenelse{\englishlanguage=1}{Outline}{Inhalt}}
			\setbeamercolor{section in toc}{fg=UniGruen,bg=}
			\setbeamercolor{section in toc shaded}{fg=Gray,bg=}
			\setbeamercolor{subsection in toc shaded}{fg=Gray,bg=}
			\setbeamercolor{subsection in toc}{fg=Gray,bg=}
			\tableofcontents[currentsection]
    \end{frame}
		\miniframeson
		\let\printpagenumbers\rememberpagenumberswitch
}
\fi

% Print table of contents at beginning of each subsection, with the current section and subsection highlighted in green and everything else shaded
\if\printtocatbeginofsubsection1
\AtBeginSubsection[]
{
			% Remove frame number from footline on outline slides
			\let\rememberpagenumberswitch\printpagenumbers
			\def\printpagenumbers{0}
		  \miniframesoff
      \begin{frame}[t,noframenumbering]{\ifthenelse{\englishlanguage=1}{Outline}{Inhalt}}
			\setbeamercolor{section in toc}{fg=UniGruen,bg=}
			\setbeamercolor{section in toc shaded}{fg=Gray,bg=}
			\setbeamercolor{subsection in toc shaded}{fg=Gray,bg=}
			\setbeamercolor{subsection in toc}{fg=UniGruen,bg=}
			\tableofcontents[currentsection,currentsubsection]
    \end{frame}
		\miniframeson
		\let\printpagenumbers\rememberpagenumberswitch
}
\fi

% Include small "Uni-Loewe" icon in top right corner of each slide
\if\printlion1
\addtobeamertemplate{frametitle}{}{%
\tikzexternaldisable%
\begin{tikzpicture}[remember picture,overlay]%
\node[anchor=north east,yshift=1.5pt, xshift = 0.2pt] at (current page.north east) {\includegraphics[height=.6cm]{figures/loewe-weiss.pdf}};%
\end{tikzpicture}%
\tikzexternalenable%

\vspace{-.5cm}
}
\fi

%%%%%%%%%%%%%%%%%%%%%%%%%%%%%%%%%%%%%%%%%%%%%%%%%%%%%%%%%%%%%%%%%%%%%%%%%
% Title page

% Do not count title page in page numbering
\let\otp\titlepage
\renewcommand{\titlepage}{\otp\addtocounter{framenumber}{-1}}

% Custom title page layout
\defbeamertemplate*{title page}{customized}
{
\thispagestyle{empty}

%\vspace{-.5cm}
\if\printbanner1
\hpcbanner
\fi

	\vspace*{.25cm}
	\begin{center}
	\if\printcoursename1
	\Large\textbf{\coursename}\par
	\fi
	\bigskip
	\hfill
	\begin{beamercolorbox}[rounded=true, center, wd=.8\paperwidth]{mycolor}
  \Large\inserttitle
  \end{beamercolorbox}
  \hfill\hfill
	
  \bigskip
	\bigskip
  
	\if\printauthor1
	\normalsize\textbf{\insertauthor}\par
	\fi
	
	\vspace{.15cm}
	\bigskip
  \if\printinstitute1
	\small\insertinstitute\par
	\fi
	\bigskip
	\if\printdate1
  \normalsize\insertdate\par
	\fi
	\ifx\longtitle\undefined
	\else
		\if\longtitle1
		\vspace{-1cm}
		\else
		
		\fi
	\fi
	\end{center}
}

% Generate HPC banner for title page, similar to our exercise sheet headers etc.
\newcommand{\hpcbanner}{
	\makebox[\textwidth][c]{
		\begin{tikzpicture}
			[textnode/.style={white,font={\bf \sffamily \small},inner sep=0pt}]
						\fill [UniGruen] (0,0) rectangle (\paperwidth,2cm);
			\node [inner sep=0pt] (loewe) at (1,1) {\includegraphics[width=1.75cm]{figures/loewe-weiss.pdf}};
			\node [textnode,anchor=west] (T) at (2.25,1) {\ifthenelse{\englishlanguage=1}{Scientific Computing \& High Performance Computing}{Wissenschaftliches Rechnen und Hochleistungsrechnen}};
			\node [textnode,above=0.6cm of T.west,anchor=west] {Bergische Universität Wuppertal};
			\node [textnode,below=0.6cm of T.west,anchor=west] {\lecturername};
		\end{tikzpicture}
	}
}

%%%%%%%%%%%%%%%%%%%%%%%%%%%%%%%%%%%%%%%%%%%%%%%%%%%%%%%%%%%%%%%%%%%%%%%%%
% Auxiliary stuff

% Remove algorithm numbering
\renewcommand{\thealgocf}{}

% Command for including small book icon that can be used for referencing literature, lecture notes or similar
\newcommand{\smallbook}{\includegraphics[width = .028\textwidth]{figures/bookicon}}

% Command that generates a framed box containing a book icon and text
\newcommand{\inbook}[1]{{\setlength{\topsep}{0pt}
\begin{framed}
\begin{minipage}{.08\textwidth}
\includegraphics[width = .99\textwidth]{figures/bookicon}
\end{minipage}
\begin{minipage}{.9\textwidth}
#1
\end{minipage}
\end{framed}}}

% Slight change of bibliography layout to look better on slides
\let\OLDthebibliography\thebibliography
\renewcommand\thebibliography[1]{
  \OLDthebibliography{#1}
  \setlength{\parskip}{0pt}
  \setlength{\itemsep}{0pt plus 0.3ex}
}

%%%%%%%%%%%%%%%%%%%%%%%%%%%%%%%%%%%%%%%%%%%%%%%%%%%%%%%%%%%%%%%%%%%%%%%%%
% Some commands that I frequently need
\newcolumntype{C}[1]{>{\centering\arraybackslash}p{#1}}
% Define your logos here to be used on the title page
\newcommand{\BUWLogo}{\includegraphics[height=32pt]{figures/buw}}
\newcommand{\upk}{^{(k)}}
\newcommand{\upm}{^{(m)}}
\newcommand{\upinv}{^{-1}}
\newcommand{\R}{\mathbb{R}}
\newcommand{\N}{\mathbb{N}}
\newcommand{\C}{\mathbb{C}}
\newcommand{\K}{\mathcal{K}}
\newcommand{\bigO}{\mathcal{O}}
\newcommand{\Rn}{\mathbb{R}^n}
\newcommand{\Rk}{\mathbb{R}^k}
\newcommand{\Cn}{\mathbb{C}^n}
\newcommand{\Rnn}{\mathbb{R}^{n \times n}}
\newcommand{\Cnn}{\mathbb{C}^{n \times n}}
\newcommand{\slitplane}{\mathbb{C} \setminus \mathbb{R}_0^-}
\newcommand{\va}{{\mathbf a}}
\newcommand{\vb}{{\mathbf b}}
\newcommand{\vc}{{\mathbf c}}
\newcommand{\vd}{{\mathbf d}}
\newcommand{\vdhat}{{\mathbf {\hat d}}}
\newcommand{\ve}{{\mathbf e}}
\newcommand{\vehat}{{\mathbf {\hat e}}}
\newcommand{\vf}{{\mathbf f}}
\newcommand{\vftilde}{{\mathbf {\widetilde f}}}
\newcommand{\vg}{{\mathbf g}}
\newcommand{\vh}{{\mathbf h}}
\newcommand{\vhhat}{{\mathbf {\hat h}}}
\newcommand{\vk}{{\mathbf k}}
\newcommand{\vp}{{\mathbf p}}
\newcommand{\vq}{{\mathbf q}}
\newcommand{\vr}{{\mathbf r}}
\newcommand{\vs}{{\mathbf s}}
\newcommand{\vt}{{\mathbf t}}
\newcommand{\vu}{{\mathbf u}}
\newcommand{\vv}{{\mathbf v}}
\newcommand{\vw}{{\mathbf w}}
\newcommand{\vx}{{\mathbf x}}
\newcommand{\vy}{{\mathbf y}}
\newcommand{\vz}{{\mathbf z}}
\newcommand{\vnull}{\boldsymbol{0}}
\newcommand{\vone}{\boldsymbol{1}}
\newcommand{\spK}{{\cal K}}
\newcommand{\spEK}{{\cal E}}
\newcommand{\spL}{{\cal L}}
\newcommand{\spP}{{\cal P}}
\newcommand{\tol}{\texttt{tol}{}}
\newcommand{\specialcell}[2][l]{\begin{tabular}[#1]{@{}c@{}}#2\end{tabular}}
\renewcommand{\d}{\,\mathrm{d}}
\newcommand{\dmu}{\d\mu(t)}
\newcommand{\dalpha}{\d\alpha(z)}
\DeclareMathOperator{\Pe}{Pe}
\DeclareMathOperator{\spec}{spec}
\DeclareMathOperator{\diag}{diag}
\DeclareMathOperator{\range}{range}
\DeclareMathOperator{\Span}{span}
\DeclareMathOperator{\trace}{trace}
\DeclareMathOperator{\tr}{tr}
\DeclareMathOperator{\sign}{sign}
\DeclareMathOperator*{\argmin}{arg\,min}
\DeclareMathOperator*{\argmax}{arg\,max}
\newcommand{\lmin}{{\lambda_{\min}}}
\newcommand{\lmax}{{\lambda_{\max}}}
\newcommand{\AHA}{A^H\!A}
\newcommand{\tve}{\widetilde{{\mathbf e}}}
\newcommand{\tvf}{\widetilde{{\mathbf f}}}
\newcommand{\tvx}{\widetilde{{\mathbf x}}}
\newcommand{\tvr}{\widetilde{{\mathbf r}}}
\newcommand{\rhoinvA}{\rho}
\newcommand{\deltaA}{\delta}
\newcommand{\deltainvA}{\delta'}
\newcommand{\Lmax}{\Lambda_{\max}}
\newcommand{\nmin}{\nu_{\min}}
\newcommand{\nmax}{\nu_{\max}}
\newcommand{\calO}{\mathcal{O}}
\newcommand{\comment}[1]{{\small\color{gray!50}// #1}}
\DeclareMathAlphabet{\mathbf}{OT1}{cmr}{bx}{n}
\def\Hat{\mkern-3mu\text{\textasciicircum}}
 

\title{Beispiel-Präsentation}
\date{11.10.2023}

\begin{document}

%%%%%%%%%%%%%%%%%%%%%%%%%%%%%%%%%%%%%%%%%%%%%%%%%%%%%%%%%%%%%%%%
% title slide
%\maketitle
%%%%%%%%%%%%%%%%%%%%%%%%%%%%%%%%%%%%%%%%%%%%%%%%%%%%%%%%%%%%%%%%


%%%%%%%%%%%%%%%%%%%%%%%%%%%%%%%%%%%%%%%%%%%%%%%%%%%%%%%%%%%%%%%%
% outline slide (without frame number in foot line)
\let\rememberpagenumberswitch\printpagenumbers
\def\printpagenumbers{0}

\begin{frame}[t,noframenumbering]{\ifthenelse{\englishlanguage=1}{Outline}{Inhalt}}
	\tableofcontents
\end{frame}
\let\printpagenumbers\rememberpagenumberswitch
%%%%%%%%%%%%%%%%%%%%%%%%%%%%%%%%%%%%%%%%%%%%%%%%%%%%%%%%%%%%%%%%


%%%%%%%%%%%%%%%%%%%%%%%%%%%%%%%%%%%%%%%%%%%%%%%%%%%%%%%%%%%%%%%%%%%%%%%%%
%
% Content starts here
%
%%%%%%%%%%%%%%%%%%%%%%%%%%%%%%%%%%%%%%%%%%%%%%%%%%%%%%%%%%%%%%%%%%%%%%%%%

\section{Dieser Beamer-Style}

\begin{frame}[t]{Dieser Beamer-Style}
	\begin{itemize}
		\item Sie können diese \LaTeX-Vorlage für das Vorbereiten ihrer Vortragsfolien verwenden.

		\item Es ist ihnen selbstverständlich freigestellt, auch eine andere Vorlage zu verwenden

		\item Die Vorlage ist in der vorliegenden Fassung so konfiguriert, dass sie sich sinnvoll für kürzere Vorträge eignet.

		\item Am Anfang der .tex-Dateien stehen einige Optionen zur Verfügung, die ich eher für Vorlesungen etc.\ verwende (beispielsweise ein Einblenden des Inhaltsverzeichnisses vor jedem Unterabschnitt). Sie können diese Elemente natürlich einschalten, wenn Sie möchten.
	\end{itemize}
\end{frame}

\begin{frame}[t]{Hinweis zum Kompilieren}
	\begin{itemize}
		\item Der Beamerstyle verwendet die Option \texttt{tikzexternalize} um die Übersetzungszeit möglichst gering zu halten

		\item Damit das funktioniert, muss das Dokument mittels
		      \begin{alltt}
			      \scriptsize
			      pdflatex -synctex=1 -interaction=nonstopmode --shell-escape \%.tex
		      \end{alltt}
		      kompiliert werden.

		\item Die erzeugten Grafiken werden im Ordner \texttt{tikzfigures} abgelegt (wird automatisch angelegt).
	\end{itemize}
\end{frame}

\section{Layout-Optionen}

\begin{frame}[t]{Konfigurierbarkeit}
	\begin{itemize}
		\item Die Beispieldatei enthält zu Beginn einige Optionen zum Anpassen des Beamerstils.

		\item Die meisten sind selbsterklärend und außerdem ausreichend kommentiert, im Folgenden werden aber die wichtigsten Optionen einmal vorgestellt.
	\end{itemize}
\end{frame}


\subsection{Metadaten}

\begin{frame}[t]{Kurs-Metadaten}
	\begin{itemize}
		\item Wenn der Vortrag Teil eines Kurses ist, kann man über die Kommandos

		      \begin{alltt}
			      \textbackslash{}newcommand\{\textbackslash{}coursename\}\{Kurs-Name\}\\
			      \textbackslash{}newcommand\{\textbackslash{}coursenamefootline\}\{Kurs-Name\}
		      \end{alltt}

		      einen Kurstitel anlegen, der auf der Titelseite und in der Fußzeile angezeigt wird.

		\item Außerdem kann über
		      \begin{alltt}
			      \textbackslash{}def\textbackslash{}englishlanguage\{0\}
		      \end{alltt}
		      die Sprache zwischen Deutsch und Englisch umgeschaltet werden. Dies beeinflußt beispielsweise das Banner auf der Titelseite sowie die Überschrift des Inhaltsverzeichnisses.
	\end{itemize}

\end{frame}

\subsection{Titelseite}

\begin{frame}[t]{Titelseite}
	\begin{itemize}
		\item Für das Design der Titelseite stehen die folgenden Switches zur Verfügung:

		      \begin{alltt}
			      \textbackslash{}def\textbackslash{}printbanner\{1\}\\
			      \textbackslash{}def\textbackslash{}printauthor\{0\}\\
			      \textbackslash{}def\textbackslash{}printdate\{1\}\\
			      \textbackslash{}def\textbackslash{}printinstitute\{0\}\\
			      \textbackslash{}def\textbackslash{}printcoursename\{1\}\\
		      \end{alltt}

		\item Diese bestimmen, welche Elemente auf der Titelseite angezeigt werden.

		\item Wenn der Vortragende und der Kursverantwortliche gleich sind, empfiehlt es sich, nur entweder das Banner oder Autor/Institut anzuzeigen, um eine Dopplung dieser Informationen zu vermeiden.
	\end{itemize}

\end{frame}

\subsection{Kopf- und Fußzeile}

\begin{frame}[t]{Kopf- und Fußzeile}
	\begin{itemize}
		\item In der Kopfzeile wird oben rechts ein Uni-Löwe angezeigt. Dieser kann per
		      \begin{alltt}
			      \textbackslash{}def\textbackslash{}printlion\{0\}
		      \end{alltt}
		      ausgeschaltet werden. \emph{(Aufgrund von tikzexternalize verschwindet der Löwe manchmal von einigen Slides, nach einem weiteren Mal kompilieren taucht er aber wieder auf)}

		\item Die Seitenzahlen in der Fußzeile können mittels
		      \begin{alltt}
			      \textbackslash{}def\textbackslash{}printpagenumbers\{0\}
		      \end{alltt}
		      ausgeschaltet werden. Auf Titelseite und Inhaltsverzeichnis werden grundsätzlich keine Seitenzahlen angezeigt.

		\item Wer unbedingt will, kann per
		      \begin{alltt}
			      \textbackslash{}def\textbackslash{}printnavigationsymbols\{1\}
		      \end{alltt}
		      die Navigations-Symbole einschalten.

	\end{itemize}

\end{frame}

\subsection{Inhaltsverzeichnisse}

\begin{frame}[t]{Inhaltsverzeichnisse}
	\begin{itemize}
		\item Um längere Vorträge zu strukturieren, kann es sinnvoll sein, zu Beginn jeder Section (oder sogar jeder Subsection) ein Inhaltsverzeichnis anzuzeigen, in dem die aktuelle Section (und gegebenenfalls Subsection) farblich hervorgehoben ist.

		\item Da dies für kurze Vorträge weniger sinnvoll ist, ist das Inhaltsverzeichnis zu Beginn der Subsection ausgeschaltet.

		\item Um diese Konfiguration anzupassen, stehen die beiden Schalter
		      \begin{alltt}
			      \textbackslash{}def\textbackslash{}printtocatbeginofsection\{1\}\\
			      \textbackslash{}def\textbackslash{}printtocatbeginofsubsection\{0\}
		      \end{alltt}
		      zur Verfügung.
	\end{itemize}

\end{frame}


\section{Slides strukturieren}

\begin{frame}[t]{Boxen und ähnliches}

	Ein paar Beispiele für Boxen und verwandte Layoutelemente

	\vspace{.5cm}
	\begin{beamerboxesrounded}{Das ist eine Box}
		Box für Definitionen, Sätze etc.
	\end{beamerboxesrounded}

	\vspace{.5cm}
	\begin{beamerboxesrounded}{Eine Box mit Literaturverweis \smallbook}
		Subtiler Verweis auf Skript o.ä.

		Wird erzeugt durch das Kommando \texttt{\textbackslash{}smallbook}
	\end{beamerboxesrounded}

	\vspace{.5cm}
	\inbook{Hier steht etwas, was man im Skript/einem Buch etc.\ genauer nachlesen kann. Wird erzeugt über das Kommando
		\begin{center}
			\texttt{\textbackslash{}inbook\{Text in der Box\}}
		\end{center}}

\end{frame}


\begin{frame}[t]{Verdeckte Inhalte}

	\begin{itemize}
		\item Es kann sinnvoll sein, Teile des Folieninhalts zunächst zu verdecken und erst nach und nach anzuzeigen, so wie es auf dieser Folie der Fall ist.

		\item<2-> Dadurch kann man die Aufmerksamkeit der Zuhörer*innen genauer auf den Punkt lenken, über den man gerade spricht.

		\item<3-> Mit Hilfe des Schalters

		      \begin{alltt}
			      \textbackslash{}def\textbackslash{}coveredtransparent\{1\}\\
		      \end{alltt}

		      kann man bestimmen, ob verdeckte Inhalte transparent dargestellt werden sollen (wie auf dieser Folie), oder vollständig unsichtbar sind.
	\end{itemize}
\end{frame}


\tikzexternaldisable
\begin{frame}[t]{TikZ-Pfeile per Overlay}

	\begin{itemize}
		\item Manchmal ist es nützlich, Verbi\tikz[na] \coordinate (anker1);ndungen auf Slides darzustellen. Dafür habe ich ein paar TikZ-Vorkehrungen getroffen:

		\item Zunächst werden mittels \texttt{\textbackslash{}tikz[na] \textbackslash{}coordinate (anker1);} und \texttt{\textbackslash{}tikz[na] \textbackslash{}coordinate (anker2);} an beliebigen Stellen auf der Slide zwei Anker für den Pfeil definiert.

		\item Anschließend können diese dann per
		      \begin{alltt}\textbackslash{}connectbyarrow\{anker1\}\{anker2\}\{Krümmung\}
		      \end{alltt}
		      verb\tikz[na] \coordinate (anker2);unden werden. Dabei akzeptiert \texttt{Krümmung} die üblichen TikZ-Spezifikationen wie z.B. \texttt{bend left=15}.

		\item \emph{Schönheitsfehler:} Frames, auf denen solche Verbindungen genutzt werden, müssen in \texttt{\textbackslash{}tikzexternaldisable} \dots \texttt{\textbackslash{}tikzexternaldisable} eingefasst werden.

	\end{itemize}

	\only<2>{
		\connectbyarrow{anker1}{anker2}{bend left=15}
	}
\end{frame}
\tikzexternalenable

\begin{frame}[t]{Algorithmen}

	\begin{itemize}
		\item Algorithmen brauchen wir natürlich auch, die Sprache wird automatisch durch den vorher erwähnten Sprach-Switch umgeschaltet.
	\end{itemize}

	\begin{algorithm}[H]
		\DontPrintSemicolon
		\KwIn{$n \in \N$}
		\BlankLine
		$a \leftarrow 1$\;
		\For{$j = 2,\dots,n$}{
			$a \leftarrow a \cdot j$\;
		}
		\If{$a > 100$}
		{
			$a \leftarrow 0$\;
		}
		\caption{Sinnloser Beispiel-Algorithmus}
	\end{algorithm}


\end{frame}

\end{document}



